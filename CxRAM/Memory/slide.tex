%
\begin{Frame}{Intro : Classical Memory}
  \begin{columns}[t]
    \begin{column}{\BW} % Colonne gauche
      \begin{block}{Memory technology}
        \begin{description}[DRAM]
        \item[\href{https://en.wikipedia.org/wiki/Dynamic_random-access_memory}{DRAM}]
          Dense, but need refresh (external)
        \item[\href{https://en.wikipedia.org/wiki/Static_random-access_memory}{SRAM}]
          Less dense but faster (caches)
        \item[Other] RRAM, STTRAM, \ldots
        \end{description}
      \end{block} 
      \begin{block}{Memory access}
        \begin{description}[LOAD]
        \item[LOAD] data or insn
          \begin{enumerate}
          \item Assert address
          \item Read data
          \end{enumerate}
        \item[STORE] data
          \begin{enumerate}
          \item Assert address \& data
          \end{enumerate}
        \end{description}
      \end{block} 
    \end{column}
    
    \begin{column}{\BW} % Colonne droite
      \begin{block}{Memory schematic}
        \Image[5cm]{Memory/ClassicalMemory.pdf}
      \end{block}   
    \end{column}
  \end{columns}  
\end{Frame}

%% Local Variables:
%% mode: latex
%% coding: utf-8
%% ispell-dictionary: "american"
%% TeX-master: "../../main.tex"
%% End:

